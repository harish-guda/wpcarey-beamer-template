\documentclass[9pt, mathserif]{beamer}
\usepackage{eulervm}
\usepackage{multirow}
\usepackage{amssymb}
\usepackage{amsxtra}
\usepackage{graphicx}
\usepackage{epsfig}
\usepackage{natbib}
\usepackage{url}
\usepackage{xcolor}
\usepackage{amsmath}
\usepackage{comment}
\usepackage{hyperref}
\usepackage{graphicx}
\usepackage{tikz}
\usepackage{comment}
\usepackage{verbatim}
\usepackage{multicol}
\beamertemplatenavigationsymbolsempty
\definecolor{asumaroon}{RGB}{128, 0, 0}
\definecolor{asugold}{RGB}{164,114,11}
\definecolor{olive}{RGB}{85,107,47}
\usecolortheme[named=asumaroon]{structure}
\def\callout#1{\textcolor{asugold}{#1}}
\def\bigquestion#1{\textcolor{asugold}{
\vspace{-1.5em}
\begin{center}
\hrulefill\\
\Large \textbf{#1}\\
\vspace{-0.5em}
\hrulefill
\end{center}
}}
\def\newterm#1{\textcolor{asugold}{\textbf{#1}}}
%For Progress Bar
\usetikzlibrary{calc}
\makeatletter
\def\progressbar@progressbar{} % the progress bar
\newcount\progressbar@tmpcounta% auxiliary counter
\newcount\progressbar@tmpcountb% auxiliary counter
\newdimen\progressbar@pbht %progressbar height
\newdimen\progressbar@pbwd %progressbar width
\newdimen\progressbar@rcircle % radius for the circle
\newdimen\progressbar@tmpdim % auxiliary dimension
\progressbar@pbwd=\linewidth
\progressbar@pbht=1pt
\progressbar@rcircle=2.5pt
% the progress bar
\def\progressbar@progressbar{%
\progressbar@tmpcounta=\insertframenumber
\progressbar@tmpcountb=\inserttotalframenumber
\progressbar@tmpdim=\progressbar@pbwd
\multiply\progressbar@tmpdim by \progressbar@tmpcounta
\divide\progressbar@tmpdim by \progressbar@tmpcountb
\begin{tikzpicture}
\draw[asumaroon!10,line width=\progressbar@pbht]
(0pt, 0pt) -- ++ (\progressbar@pbwd,0pt);
\filldraw[asumaroon] %
(\the\dimexpr\progressbar@tmpdim-\progressbar@rcircle\relax, .5\progressbar@pbht) circle (\progressbar@rcircle);
\node[draw=asumaroon,text width=3.5em,align=center,inner sep=1pt,
text=asumaroon,anchor=east] at (0,0) {\insertframenumber/\inserttotalframenumber};
\end{tikzpicture}%
}
\addtobeamertemplate{footline}{}
{%
  \begin{beamercolorbox}[wd=\paperwidth,ht=4ex,center,dp=1ex]{violetHG}%
    \progressbar@progressbar%
  \end{beamercolorbox}%
}
\makeatother
%End of Progress Bar
\usebackgroundtemplate{
  \tikz[overlay,remember picture]
  \node[opacity=1, at=(current page.south east),anchor=south east,inner sep=0.50em] {
    \includegraphics[height=0.75cm,width=1.75cm]{asu-wpc-logo.JPG}};
}
%End of Watermark

\makeindex

\begin{document}

\setbeamercovered{transparent}

\begin{frame}

\title{\LARGE{Main Title}}
\subtitle{\Large{Subtitle}}
\author[HG, x, y]{Harish Guda\inst{1} \and Author x\inst{2}}
\institute
{
\inst{1}
Department of Supply Chain Management \\
W.P. Carey School of Business \\
Arizona State University
\and
\inst{2}
Department of Operations Management \\
Jindal School of Management \\
The University of Texas at Dallas
}
\date[lecture]{Seminar on Pricing and Revenue Management, \\
\today}

\maketitle
\end{frame}

%The Summary Slide

\begin{frame}

\frametitle{A Brief Summary}

\begin{itemize}[<+->]
\setlength\itemsep{2em}
\item Summary points

\item One key term: \newterm{bid-price control}.

\item Some approximations.

\end{itemize}
\end{frame}

\AtBeginSection[]
{
 \begin{frame}
 \frametitle{Agenda}
 \tableofcontents[currentsection]
 \addtocounter{framenumber}{-1}
 \end{frame}
}

\section{Introduction}

\begin{frame}
\frametitle{Introduction}
\begin{itemize}[<+->]\setlength\itemsep{2em}

\item \newterm{Key term}: Definition.

\item Customers buy bundles of resources in combination.
\begin{itemize}
\setlength\itemsep{1em}
\item Example 1.
\end{itemize}
\end{itemize}\pause
\bigquestion{Main Research Question: What is the role of $\delta$ on $\gamma$?}
\end{frame}

\section{Model}

\begin{frame}
\frametitle{Model}
\begin{itemize}[<+->]\setlength\itemsep{2em}
\item Key parameters of agent $W$: $\theta$, $\gamma$

\item Key parameters of agent $C$: $\delta$, $\omega$.

\end{itemize}
\end{frame}

\section{Results}

\begin{frame}
\frametitle{Key Non-Existence Result}
\begin{itemize}[<+->]\setlength\itemsep{2em}
\item Here is a result. 
\end{itemize}

\bigskip

\begin{theorem}
Suppose $\gamma > 0$. There does not exist an outcome with $\delta > 0$ and $\Delta > 0$. That is,
\begin{equation}
\gamma > 0 \implies \delta \cdot \Delta < 0.
\end{equation}
\end{theorem}


\end{frame}

\begin{frame}
\begin{center}
\Huge \textcolor{asumaroon}{Thank You.}
\end{center}

\end{frame}

\end{document}
